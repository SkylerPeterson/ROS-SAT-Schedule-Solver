\documentclass{article}
\title{Optimal Dub-E Scheduling}
\author{Skyler Peterson, Alex Sanchez-Stern}

\begin{document}
\maketitle
\section{Introduction}
For our final project,
we decided to look to robotics
for problems that SMT might be able to tackle.
Since we are fortunate enough to be at a school
where there is always interesting work going on across fields,
we didn't have to look far.
We contacted Michael Jae-Yoon Chung and Andrzej Pronobis,
who are working on the Semantics Aware Robotic Assistant,
more commonly known as DUB-E.
DUB-E is able to traverse the CSE building,
and accomplish tasks for it's users,
such as checking whether a particular professor is in their office.
DUB-E is controlled via a web interface,
from which users can request that certain tasks be accomplished,
by a particular deadline.
Unfortunately, when deadlines are short
and there are many tasks to accomplish,
it is non-trivial to decide what task
should be accomplished when.
Additionally, the expressive power of the interface to DUB-E
is currently limited;
Users can specify a tasks deadline,
but cannot specify more precise timing information,
such as a time in the future
before which the task should not be done,
or multiple time periods in which a task can be accomplished.
\section{Overview}
\section{Scheduler Encoding}
\section{Results}
\section{Project Division}
We roughly divided the work into the encoding,
and the interface with DUB-E.
Skyler handled the interface with the robot,
setting up ROS nodes and working with Michael and Andrzej
to get the encoder integrated
into the DUB-E codebase.
Alex handled the actual encoding of the task constraints,
setting up the Z3 bindings to interface with the code,
and writing modules to take in task information,
encode it into a set of Z3 constraints,
decode the Z3 output into a schedule,
and format the schedule for DUB-E.
\section{Applied Topics}
This project involved mostly topics
that we discussed at the beginning of the quarter,
although it of course was supported by
the work we did throughout the quarter.
Specifically, the work on encoding different types of constraints
into conjunctive normal form was paramount to the project.
The encoding also made heavy use of MaxSAT,
which we touched on in class,
to satisfy the greatest number of tasks
in cases where not all tasks could be satisfied.
\end{document}
